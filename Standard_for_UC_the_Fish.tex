\documentclass{article}
\usepackage{graphicx}
\usepackage[colorlinks=false]{hyperref}

\begin{document}

\title{Hacking 10Base--T Ethernet\\for Underwater Optical Communication}
\author{UC the Fish}

\maketitle

\section*{Introduction}

One of the funtional requirements for UC the Fish was modular optic communication,
or the communication system should be easy to integrate with devices that need to
communicate; a plug--and--play solution.

Nearly every modern computer has at least \mbox{10Base--T Ethernet} and
\mbox{USB 2.0} connectivity, so any ``modular'' communication attachement
should use one of these protocols--at least at its endpoints.
We chose to \mbox{10Base--T ethernet}, not just at the interfaces but to
send the ethernet signal itself through water in the form of intensity
of blue light.
This allowed us to focus on building blue light transmit and receive
hardware instead of attempting to develop light transmit/receive hardware
\textit{and} a modulation protocol + logic giving it USB endpoints.

Ethernet allows multiple stations to share a single medium,
which is appropriate for water because light spreads in every direction
and data cannot be multiplexed over different wavelegth channels. 
It includes hardware support for cyclic redundancy bit checking,
up to 16 retransmission attempts in the case of bit errors or disruption
of the medium, and at \mbox{10 MHz}, ethernet is fast enough 
for live streaming video.

Unlike USB, a temporary dissruption in the data link does not require
renegotiation time and application level link management.
Software applications can use the operating system to handle TCP protocol
when data integrity and delivery aknowledgement are necessary, instead
of writing custom code to ensure control commands reach the sub intact.

\section*{Getting the Standard}

IEEE Std 802.3 is the standard for Ethernet communication.
It is freely available online but is so large it is split into multiple
sections.
The first section (only 555 pages!) is available at
\url{https://standards.ieee.org/getieee802/download/802.3-2012\_section1.pdf}.

\section*{Ethernet 101 \footnote{Or in other words, an Ethernet Preamble. Hehe.}}

\section*{More Details from the Standard}



\section*{Influence on Transmitter/Receiver Design}

\section*{Testing a Transmit/Receive Pair}

\end{document}
